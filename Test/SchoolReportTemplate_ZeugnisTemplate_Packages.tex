% ----------------------------------------------------
%  Minified, LuaLaTeX-friendly preamble for “Zeugnis”
% ----------------------------------------------------
\documentclass[12pt,a4paper]{article}

% ---------------- Page layout & colours -------------
\usepackage[top=2cm,bottom=2cm,
inner=1.5cm,outer=1.5cm]{geometry}

\usepackage[table]{xcolor}
\definecolor{egeblue}{RGB}{0,99,142}
\definecolor{egelightblue}{RGB}{66,144,179}
\definecolor{greenEnglish}{rgb}{0,0.5,0}

% ---------------- Language --------------------------
\usepackage[german]{babel}      % LuaLaTeX handles UTF-8 natively

% ---------------- Fonts -----------------------------
\usepackage{fontspec}           % modern font loader
\setmainfont{Latin Modern Sans}        % or any installed font
%\setsansfont{Helvetica Neue}            % replaces \usepackage{helvet}
%\setsansfont{Latin Modern Sans}
\usepackage{amsmath,amssymb}
%\usepackage{unicode-math}       % full Unicode maths
%\setmathfont{Latin Modern Math}

% If you need one single sans-serif default:
% \renewcommand*\familydefault{\sfdefault}

% ---------------- Micro-typography ------------------
\usepackage{microtype}          % protrusion & font expansion
\usepackage{selnolig}           % fixes spurious ligatures

% ---------------- Headings, spacing -----------------
\usepackage{titlesec}
\usepackage{setspace}           % \singlespacing, etc.
\usepackage{ragged2e}
\usepackage{enumitem}

% ---------------- Lua helpers -----------------------
\usepackage{luacode}

% ---------------- Graphics / header / footer --------
\usepackage{graphicx}
\usepackage{fancyhdr}

% ---------------- Tables & boxes --------------------
\usepackage{array,tabularx,xltabular}
%\setlength{\LTpre}{0pt}\setlength{\LTpost}{0pt}
\usepackage{cellspace}
\makeatletter
\let\@startpbox@action\@startpbox
\makeatother

\usepackage{makecell,colortbl,booktabs,multirow}

\usepackage{nowidow}            % prevents widows/orphans

% convenient column types
\newcolumntype{Y}{>{\centering\arraybackslash}X}
\newcolumntype{L}[1]{>{\raggedright\arraybackslash}p{#1}}
%\newcolumntype{C}[1]{>{\centering\arraybackslash}m{#1}}
\newcolumntype{R}[1]{>{\raggedleft\arraybackslash}p{#1}}
\newcolumntype{J}[1]{>{\justifying\arraybackslash}p{#1}}
\newcolumntype{v}[1]{>{\raggedright\hspace{0em}}p{#1}}
\newcolumntype{C}[1]{>{\centering\arraybackslash}m{#1}}
\renewcommand*\tabularxcolumn[1]{>{}m{#1}}

\newcommand*\rot{\rotatebox{90}}
%\renewcommand\theadalign{l}
\renewcommand\theadfont{\bfseries\footnotesize}

% ---------------- Maths & symbols -------------------
\usepackage{amsmath,amssymb}

% ---------------- Strikeout, underline --------------
\usepackage[normalem]{ulem}      % keep \emph italic, enable \sout

% ---------------- Misc helpers ----------------------
\usepackage{xstring,etoolbox}

% If you still need a *specific* Unicode symbol that your fonts
% don’t contain, you can define it once, e.g.:
% \newunicodechar{⟨}{\ensuremath\langle}

% ----------------------------------------------------
\setlength{\parindent}{0pt}
\newcommand{\tableGradeWidth}{.080\linewidth}
\newcommand{\tableGradeWidthLevel}{.06\linewidth}
