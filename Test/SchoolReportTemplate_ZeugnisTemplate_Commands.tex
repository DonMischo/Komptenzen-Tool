\newcommand{\myZeugnisTable}[2]{
	\begin{tblr}{
			width = \linewidth,
			colspec = {Q[l,m,wd=11cm] X[c] X[c] X[c] X[c]},
			hlines = {1pt,solid},
			vlines = {0.6pt,solid},
			rowsep = 2pt,
			colsep = 3pt,
			row{1} = {bg=egelightblue, fg=white, font=\bfseries\footnotesize},
		}
		\textbf{\large #1} & 
		\shortstack[c]{\textbf{sehr gut}\\ \textbf{erfüllt}} &
		\shortstack[c]{\textbf{gut}\\ \textbf{erfüllt}} &
		\shortstack[c]{\textbf{teilweise}\\ \textbf{erfüllt}} &
		\shortstack[c]{\textbf{nicht}\\ \textbf{erfüllt}} \\
		#2
	\end{tblr}
}

\makeatletter
\def\nobreakhline{%
	\noalign{\ifnum0=`}\fi
	\penalty\@M
	\futurelet\@let@token\LT@@nobreakhline}
\def\LT@@nobreakhline{%
	\ifx\@let@token\hline
	\global\let\@gtempa\@gobble
	\gdef\LT@sep{\penalty\@M\vskip\doublerulesep}% <-- change here
	\else
	\global\let\@gtempa\@empty
	\gdef\LT@sep{\penalty\@M\vskip-\arrayrulewidth}% <-- change here
	\fi
	\ifnum0=`{\fi}%
	\multispan\LT@cols
	\unskip\leaders\hrule\@height\arrayrulewidth\hfill\cr
	\noalign{\LT@sep}%
	\multispan\LT@cols
	\unskip\leaders\hrule\@height\arrayrulewidth\hfill\cr
	\noalign{\penalty\@M}%
	\@gtempa}
\makeatother


%\newcommand\competencyLevel{
%	\begin{tabularx}{\linewidth}{|X|C{\tableGradeWidth}|C{\tableGradeWidth}|C{\tableGradeWidth}|C{\tableGradeWidth}|}
%		\hline
%		\textbf{Kompetenzfelder und Lernziele der Fachbereiche} & \rot{\parbox{2.7cm}{\thead{sehr gut erfüllt}}} & \rot{\thead{gut erfüllt}} & \rot{\thead{teilweise}} & \rot{\thead{nicht erfüllt}} \\
%		\hline
%	\end{tabularx}
%}

\newcommand{\gradeOne}{& $\boxtimes$ & $\Box$ & $\Box$ & $\Box$\\}
\newcommand{\gradeTwo}{& $\Box$ & $\boxtimes$ & $\Box$ & $\Box$\\\nobreakhline}
\newcommand{\gradeThree}{& $\Box$ & $\Box$ & $\boxtimes$ & $\Box$\\\nobreakhline}
\newcommand{\gradeFour}{& $\Box$ & $\Box$ & $\Box$ & $\boxtimes$\\\nobreakhline}
\newcommand{\gradeNotGiven}{& \multicolumn{4}{c|}{nicht erteilt}\\\nobreakhline}
\newcommand{\gradeComesWithSecondHalfYear}{& \multicolumn{4}{c|}{wird im 2. Halbjahr belegt}\\\nobreakhline}
%\newcommand{\gradeDefault}{&  &  &  & \\\nobreakhline}
\newcommand\gradeDefault{& \multicolumn{4}{c|}{\textemdash} \\}
\newcommand{\gradeSkip}{& $\Box$ & $\Box$ & $\Box$ & $\Box$\\\nobreakhline}


\newcommand{\levelOne}{\centering Du hast vorwiegend auf {\color{greenEnglish} Anforderungsebene 1} gearbeitet.}
\newcommand{\levelTwo}{\centering Du hast vorwiegend auf {\color{blue} Anforderungsebene 2} gearbeitet.}
\newcommand{\levelThree}{\centering Du hast vorwiegend auf {\color{red} Anforderungsebene 3} gearbeitet.}
\newcommand{\levelSeven}{\centering bis Klasse 7 ohne Anforderungsebene}
\newcommand{\levelEight}{\centering bis Klasse 8 ohne Anforderungsebene}
\newcommand{\levelNine}{\centering bis Klasse 9 ohne Anforderungsebene}
\newcommand{\noLevel}{}
%\newcommand{\compText}[1]{\multicolumn{5}{|p{0.975\linewidth}|}{#1}\\\hline}


% argument is the page threshold until a newline should be added
\newcommand\newOptionalNewPage[1]{
	\ifnum #1>\value{page}
		\newpage
		\thispagestyle{plain}
		\phantom{~}
	\fi
}

\newcommand{\newpagedefs}{
	\newpage
	\newgeometry{headheight=90pt,top=7cm,
		bottom=2cm, 
		inner=2cm,
		outer=2cm}
	\setlength{\headsep}{-0.4cm}
	\pagestyle{fancy}
}

\newcommand{\newpagedefsLastPage}{
	\newpage
	\newgeometry{top=2cm,
		bottom=2cm, 
		inner=2cm,
		outer=2cm}
	\pagestyle{fancy}
}

\newcommand\formatText[1]{
	\noexpandarg
	\StrBefore{#1}{\\}[\studentIntro]
	\StrBehind{#1}{\\}[\certText]
	{\Large\setstretch{1.10}\textbf{\LARGE\studentIntro\vspace{.5em}\\}\nowidow[11]\noclub[11]\certText\par}
}


\pagestyle{fancy}

\fancypagestyle{mypagestyle}{%
	\chead{\competencyLevel}
	\cfoot{\small\raggedright\textit{Legende}:	Anforderungsebene Grün = Hauptschule ( AE I ), Anforderungsebene Blau = Realschule ( AE II ), Anforderungsebene Rot = Gymnasium ( AE III ),	n.b. = nicht bewertet}
}


\widowpenalty10000
\clubpenalty10000