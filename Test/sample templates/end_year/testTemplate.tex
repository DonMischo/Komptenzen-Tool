% ----------------------------------------------------
%  Minified, LuaLaTeX-friendly preamble for “Zeugnis”
% ----------------------------------------------------
\documentclass[12pt,a4paper]{article}

% ---------------- Page layout & colours -------------
\usepackage[top=2cm,bottom=2cm,
inner=1.5cm,outer=1.5cm]{geometry}

\usepackage[table]{xcolor}
\definecolor{egeblue}{RGB}{0,99,142}
\definecolor{egelightblue}{RGB}{66,144,179}
\definecolor{greenEnglish}{rgb}{0,0.5,0}

% ---------------- Language --------------------------
\usepackage[german]{babel}      % LuaLaTeX handles UTF-8 natively

% ---------------- Fonts -----------------------------
\usepackage{fontspec}           % modern font loader
\setmainfont{Latin Modern Sans}        % or any installed font
%\setsansfont{Helvetica Neue}            % replaces \usepackage{helvet}
%\setsansfont{Latin Modern Sans}
\usepackage{amsmath,amssymb}
%\usepackage{unicode-math}       % full Unicode maths
%\setmathfont{Latin Modern Math}

% If you need one single sans-serif default:
% \renewcommand*\familydefault{\sfdefault}

% ---------------- Micro-typography ------------------
\usepackage{microtype}          % protrusion & font expansion
\usepackage{selnolig}           % fixes spurious ligatures

% ---------------- Headings, spacing -----------------
\usepackage{titlesec}
\usepackage{setspace}           % \singlespacing, etc.
\usepackage{ragged2e}
\usepackage{enumitem}

% ---------------- Lua helpers -----------------------
\usepackage{luacode}

% ---------------- Graphics / header / footer --------
\usepackage{graphicx}
\usepackage{fancyhdr}

% ---------------- Tables & boxes --------------------
\usepackage{array,tabularx,xltabular}
%\setlength{\LTpre}{0pt}\setlength{\LTpost}{0pt}
\usepackage{cellspace}
\makeatletter
\let\@startpbox@action\@startpbox
\makeatother

\usepackage{makecell,colortbl,booktabs,multirow}

\usepackage{nowidow}            % prevents widows/orphans

% convenient column types
\newcolumntype{Y}{>{\centering\arraybackslash}X}
\newcolumntype{L}[1]{>{\raggedright\arraybackslash}p{#1}}
%\newcolumntype{C}[1]{>{\centering\arraybackslash}m{#1}}
\newcolumntype{R}[1]{>{\raggedleft\arraybackslash}p{#1}}
\newcolumntype{J}[1]{>{\justifying\arraybackslash}p{#1}}
\newcolumntype{v}[1]{>{\raggedright\hspace{0em}}p{#1}}
\newcolumntype{C}[1]{>{\centering\arraybackslash}m{#1}}
\renewcommand*\tabularxcolumn[1]{>{}m{#1}}

\newcommand*\rot{\rotatebox{90}}
%\renewcommand\theadalign{l}
\renewcommand\theadfont{\bfseries\footnotesize}

% ---------------- Maths & symbols -------------------
\usepackage{amsmath,amssymb}

% ---------------- Strikeout, underline --------------
\usepackage[normalem]{ulem}      % keep \emph italic, enable \sout

% ---------------- Misc helpers ----------------------
\usepackage{xstring,etoolbox}

% If you still need a *specific* Unicode symbol that your fonts
% don’t contain, you can define it once, e.g.:
% \newunicodechar{⟨}{\ensuremath\langle}

% ----------------------------------------------------
\setlength{\parindent}{0pt}
\newcommand{\tableGradeWidth}{.080\linewidth}
\newcommand{\tableGradeWidthLevel}{.06\linewidth}


\pagenumbering{gobble}
\setlength\parindent{0pt}
\setlength\cellspacetoplimit{8pt}
\setlength\cellspacebottomlimit{8pt}

\newcommand{\tableGradeWidth}{.050\linewidth}

\newcommand{\myZeugnisTable}[2]{
	\begin{tblr}{
			width = \linewidth,
			colspec = {Q[l,m,wd=11cm] X[c] X[c] X[c] X[c]},
			hlines = {1pt,solid},
			vlines = {0.6pt,solid},
			rowsep = 2pt,
			colsep = 3pt,
			row{1} = {bg=egelightblue, fg=white, font=\bfseries\footnotesize},
		}
		\textbf{\large #1} & 
		\shortstack[c]{\textbf{sehr gut}\\ \textbf{erfüllt}} &
		\shortstack[c]{\textbf{gut}\\ \textbf{erfüllt}} &
		\shortstack[c]{\textbf{teilweise}\\ \textbf{erfüllt}} &
		\shortstack[c]{\textbf{nicht}\\ \textbf{erfüllt}} \\
		#2
	\end{tblr}
}

\makeatletter
\def\nobreakhline{%
	\noalign{\ifnum0=`}\fi
	\penalty\@M
	\futurelet\@let@token\LT@@nobreakhline}
\def\LT@@nobreakhline{%
	\ifx\@let@token\hline
	\global\let\@gtempa\@gobble
	\gdef\LT@sep{\penalty\@M\vskip\doublerulesep}% <-- change here
	\else
	\global\let\@gtempa\@empty
	\gdef\LT@sep{\penalty\@M\vskip-\arrayrulewidth}% <-- change here
	\fi
	\ifnum0=`{\fi}%
	\multispan\LT@cols
	\unskip\leaders\hrule\@height\arrayrulewidth\hfill\cr
	\noalign{\LT@sep}%
	\multispan\LT@cols
	\unskip\leaders\hrule\@height\arrayrulewidth\hfill\cr
	\noalign{\penalty\@M}%
	\@gtempa}
\makeatother


%\newcommand\competencyLevel{
%	\begin{tabularx}{\linewidth}{|X|C{\tableGradeWidth}|C{\tableGradeWidth}|C{\tableGradeWidth}|C{\tableGradeWidth}|}
%		\hline
%		\textbf{Kompetenzfelder und Lernziele der Fachbereiche} & \rot{\parbox{2.7cm}{\thead{sehr gut erfüllt}}} & \rot{\thead{gut erfüllt}} & \rot{\thead{teilweise}} & \rot{\thead{nicht erfüllt}} \\
%		\hline
%	\end{tabularx}
%}

\newcommand{\gradeOne}{& $\boxtimes$ & $\Box$ & $\Box$ & $\Box$\\}
\newcommand{\gradeTwo}{& $\Box$ & $\boxtimes$ & $\Box$ & $\Box$\\\nobreakhline}
\newcommand{\gradeThree}{& $\Box$ & $\Box$ & $\boxtimes$ & $\Box$\\\nobreakhline}
\newcommand{\gradeFour}{& $\Box$ & $\Box$ & $\Box$ & $\boxtimes$\\\nobreakhline}
\newcommand{\gradeNotGiven}{& \multicolumn{4}{c|}{nicht erteilt}\\\nobreakhline}
\newcommand{\gradeComesWithSecondHalfYear}{& \multicolumn{4}{c|}{wird im 2. Halbjahr belegt}\\\nobreakhline}
%\newcommand{\gradeDefault}{&  &  &  & \\\nobreakhline}
\newcommand\gradeDefault{& \multicolumn{4}{c|}{\textemdash} \\}
\newcommand{\gradeSkip}{& $\Box$ & $\Box$ & $\Box$ & $\Box$\\\nobreakhline}


\newcommand{\levelOne}{\centering Du hast vorwiegend auf {\color{greenEnglish} Anforderungsebene 1} gearbeitet.}
\newcommand{\levelTwo}{\centering Du hast vorwiegend auf {\color{blue} Anforderungsebene 2} gearbeitet.}
\newcommand{\levelThree}{\centering Du hast vorwiegend auf {\color{red} Anforderungsebene 3} gearbeitet.}
\newcommand{\levelSeven}{\centering bis Klasse 7 ohne Anforderungsebene}
\newcommand{\levelEight}{\centering bis Klasse 8 ohne Anforderungsebene}
\newcommand{\levelNine}{\centering bis Klasse 9 ohne Anforderungsebene}
\newcommand{\noLevel}{}
%\newcommand{\compText}[1]{\multicolumn{5}{|p{0.975\linewidth}|}{#1}\\\hline}


% argument is the page threshold until a newline should be added
\newcommand\newOptionalNewPage[1]{
	\ifnum #1>\value{page}
		\newpage
		\thispagestyle{plain}
		\phantom{~}
	\fi
}

\newcommand{\newpagedefs}{
	\newpage
	\newgeometry{headheight=90pt,top=7cm,
		bottom=2cm, 
		inner=2cm,
		outer=2cm}
	\setlength{\headsep}{-0.4cm}
	\pagestyle{fancy}
}

\newcommand{\newpagedefsLastPage}{
	\newpage
	\newgeometry{top=2cm,
		bottom=2cm, 
		inner=2cm,
		outer=2cm}
	\pagestyle{fancy}
}

\newcommand\formatText[1]{
	\noexpandarg
	\StrBefore{#1}{\\}[\studentIntro]
	\StrBehind{#1}{\\}[\certText]
	{\Large\setstretch{1.10}\textbf{\LARGE\studentIntro\vspace{.5em}\\}\nowidow[11]\noclub[11]\certText\par}
}


\pagestyle{fancy}

\fancypagestyle{mypagestyle}{%
	\chead{\competencyLevel}
	\cfoot{\small\raggedright\textit{Legende}:	Anforderungsebene Grün = Hauptschule ( AE I ), Anforderungsebene Blau = Realschule ( AE II ), Anforderungsebene Rot = Gymnasium ( AE III ),	n.b. = nicht bewertet}
}


\widowpenalty10000
\clubpenalty10000

\newcommand{\schoolReportDay}{11.02.2022}
\begin{document}
	\pagestyle{plain}
	\fcolorbox{egeblue}{white}{
	\begin{minipage}[t][.98\textheight][t]{.97\textwidth}
			~\vspace{2cm}\\
			\begin{tabularx}{\linewidth}{@{}lX}
				Versäumnisse: & 1  Tage (davon 1 Tage unentschuldigt) \\
				&  2 Stunden (davon 1 Stunden unentschuldigt) \\
			\end{tabularx}
			\vspace{3cm}\\
			Erfurt, den \schoolReportDay
			\vspace{2.5cm}\\
			\begin{tabularx}{\linewidth}{C{.4\linewidth}XC{.4\linewidth}}
				\rule{\linewidth}{1pt} &  & \rule{\linewidth}{1pt} \\
				\centering KlassenleiterIn &  & KlassenleiterIn \\
			\end{tabularx}
			\vspace{2.5cm}\\
			\begin{tabularx}{\linewidth}{C{.4\linewidth}YC{.4\linewidth}}
				\rule{\linewidth}{1pt} &  &  \\
				\centering SchuleiterIn & \tiny Siegel & \\
			\end{tabularx}
			\vspace{2.5cm}\\
			\begin{tabularx}{\linewidth}{C{.4\linewidth}XC{.4\linewidth}}
			\rule{\linewidth}{1pt} &  & \rule{\linewidth}{1pt} \\
			\centering Erziehungsberechtigte &  & SchülerIn \\
			\end{tabularx}
		\end{minipage}
	}%
	\newpage
	\fcolorbox{egeblue}{white}{
		\begin{minipage}[t][.98\textheight][t]{.97\textwidth}
		\begin{center}\vspace{1cm}
			\includegraphics[width=.9\linewidth]{Logotop_mit_ESM_zeugnis}
		\end{center}
		\vspace{3cm}
		\begin{center}\setstretch{1.5}
			\textbf{\Huge Zeugnis}\\
			\textbf{\LARGE ~\\
			Evangelischen Gemeinschaftsschule Erfurt\\\vspace{2cm}
			Otto Karl}
		\end{center}\vspace{3cm}
		\begin{center}\LARGE
			\begin{tabular}{ll}
			\textbf{Klasse:} &  1\vspace{1cm}\\
			\textbf{Name:} &  2019/2020 5a\\
			& geb.: 27.12.85
		\end{tabular}
		\end{center}
	\end{minipage}
	}%
	\newpagedefs

	\competencyTableMajorSubject{Deutsch}{\competencyMajorSubject{\textbf{Über Sprache, Sprachverwendung und Sprachenlernen reflektieren}\vspace{.3em}\newline
\underline{Wortebene}\vspace{.2em}\newline
Ich kann einen Grundbestand an Rechtschreibregeln sicher anwenden.\vspace{.2em}\newline
Ich kann die Wortarten benennen.\vspace{.2em}\newline
Ich kann Rechtschreibstrategien erkennen und sicher anwenden.}{\gradeTwo}
}{rot}
\competencyTableMajorSubject{Mathe}{\competencyMajorSubject{Rechenoperationen\vspace{.2em}\newline
Ich kann die Grundrechenoperationen im Bereich der natürlichen im Kopf und schriftlich ausführen und an Beispielen den Zusammenhang zwischen Rechenoperationen und deren Umkehroperationen erläutern. \vspace{.2em}\newline
Ich kann Teiler und Vielfache natürlicher Zahlen bestimmen.\vspace{.2em}\newline
Ich kann ein Verfahren zur Bestimmung von Primzahlen anwenden.}{\gradeFour}
\competencyMajorSubject{Mathematik mit Alltagsbezug\\
Ich kann Größen der Zeit, der Länge, der Masse, des Geldes, vergleichen, ordnen und umrechnen.\vspace{.2em}\newline
Ich kann einfache Probleme aus dem Alltag lösen, in denen mehrere Rechenoperationen miteinander zu verknüpfen sind und negative Zahlen vorkommen (z. B. Temperaturänderungen).}{\gradeThree}
\competencyMajorSubject{Rationale Zahlen\vspace{.2em}\newline
Ich kann natürliche und gebrochene Zahlen in verschiedenen Situationen lesen, im mündlichen und schriftlichen Sprachgebrauch sicher und sachgemäß verwenden. \vspace{.2em}\newline
Ich kann Bruchteile zeichnerisch darstellen, aus geometrischen Darstellungen ablesen, gebrochene Zahlen der Situation angemessen darstellen. \vspace{.2em}\newline
Ich kann natürliche Zahlen und einfache gemeine Brüche aus Alltagssituationen ordnen und vergleichen.}{\gradeThree}
\competencyMajorSubject{Geometrie\\
Ich kann die Begriffe Strecke, Strahl, Gerade unterscheiden.\vspace{.2em}\newline
Ich kann die Lagebeziehung von Geraden beschreiben.}{\gradeTwo}
}{rot}


	\thispagestyle{mypagestyle}
\end{document}