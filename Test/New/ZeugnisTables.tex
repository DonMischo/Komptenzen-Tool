\begin{luacode*}
	-- tiny helper: escape TeX specials that may appear in the data
	local esc = {
		["\\"]="\\textbackslash{}", ["%%"]="\\%",
		["_"]="\\_", ["{"]="\\{", ["}"]="\\}",
		["&"]="\\&",  ["#"]="\\#", ["^"]="\\textasciicircum{}",
		["~"]="\\textasciitilde{}"
	}
	local function tex_escape(str)
	return (str:gsub("[\\%%_%{%}&#%^~]", esc))
	end
	
	-- check/uncheck boxes for the 4-step grading scale
	local gradeCmd = {
		[1]  = "& $\\boxtimes$ & $\\Box$      & $\\Box$      & $\\Box$\\\\",
		[2]  = "& $\\Box$      & $\\boxtimes$ & $\\Box$      & $\\Box$\\\\",
		[3]  = "& $\\Box$      & $\\Box$      & $\\boxtimes$ & $\\Box$\\\\",
		[4]  = "& $\\Box$      & $\\Box$      & $\\Box$      & $\\boxtimes$\\\\",
		ne   = "\\SetCell[c=5]{c} nicht erteilt\\\\",
		HJ2  = "\\SetCell[c=5]{c} wird im 2.~Halbjahr belegt\\\\",
		default = "\\SetCell[c=5]{c} ~\\\\"
	}
	
	-- map a level (number or free text) to the line that goes into column 5
	local function levelLine(level)
		-- numeric mapping 1–3, 7–9
		if type(level) == "number" then
			local tbl = {
				[1] = "\\SetCell[c=5]{c} Du hast vorwiegend auf {\\color{greenEnglish} Anforderungsebene~1} gearbeitet.\\\\",
				[2] = "\\SetCell[c=5]{c} Du hast vorwiegend auf {\\color{blue} Anforderungsebene~2} gearbeitet.\\\\",
				[3] = "\\SetCell[c=5]{c} Du hast vorwiegend auf {\\color{red} Anforderungsebene~3} gearbeitet.\\\\",
				[7] = "\\SetCell[c=5]{c} bis~Klasse~7 ohne Anforderungsebene\\\\",
				[8] = "\\SetCell[c=5]{c} bis~Klasse~8 ohne Anforderungsebene\\\\",
				[9] = "\\SetCell[c=5]{c} bis~Klasse~9 ohne Anforderungsebene\\\\",
			}
			return tbl[level] or ""          -- unknown → print nothing (nolevel)
		end
		
		-- free-text alternative: put the given text left-aligned in column 5
		if type(level) == "string" and level ~= "" then
			return "\\SetCell[c=5]{l} "..tex_escape(level).."\\\\"
		end
		
		return ""                          -- nil / empty string  =  nolevel
	end

	
	-- loop over all subjects → topics → competences
	for _, subj in ipairs(student.subjects) do
	tex.print("\\myZeugnisTable{"..tex_escape(subj.name).."}{%")
		
		for _, topic in ipairs(subj.topics) do
			tex.print("\\textbf{"..tex_escape(topic.title).."}\\par%")
			for _, comp in ipairs(topic.competences) do
				tex.print(tex_escape(comp.description).."\\par%")
			end
			-- insert the tick-box line that matches the grade
			tex.print( gradeCmd[topic.grade] or gradeCmd.default )
		end
		
		tex.print( levelLine(subj.level) )   -- or topic.level, whatever field you use
		
		tex.print("}\\par\\vspace{1em}")
	end
\end{luacode*}
