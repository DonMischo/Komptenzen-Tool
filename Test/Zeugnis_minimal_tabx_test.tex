\documentclass{article}

% ----------------- packages you already use -----------------
\usepackage{array,   colortbl, makecell}
\usepackage{tabularx,xltabular}    % ← X columns + page breaks
\usepackage[table]{xcolor}         % ← cell colours

% ----------------- helper settings --------------------------
\definecolor{egelightblue}{RGB}{219,233,255}

\newlength{\tableGradeWidth}
\setlength{\tableGradeWidth}{18mm} % just for the demo

\newcolumntype{C}[1]{>{\centering\arraybackslash}m{#1}}
\newcolumntype{Y}{>{\raggedright\arraybackslash}X} % helper
\setlength\tabcolsep{5pt} % default spacing

\begin{document}
	
	\begin{xltabular}{\linewidth}{|>{\raggedright\arraybackslash\cellcolor{egelightblue}}m{\dimexpr\linewidth-4\tableGradeWidth-5\tabcolsep-5\arrayrulewidth}|%
			C{\tableGradeWidth}|C{\tableGradeWidth}|%
			C{\tableGradeWidth}|C{\tableGradeWidth}|}
		\hline
		\textit{\textbf{Mathe}} &
		\thead{\footnotesize sehr gut\\erfüllt} &
		\thead{\footnotesize gut\\erfüllt} &
		\thead{\footnotesize teilweise\\erfüllt} &
		\thead{\footnotesize nicht\\erfüllt} \\
		\hline
	\end{xltabular}
	
\end{document}
