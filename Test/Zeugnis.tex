% --------------------------------------------------------------
%  Simple LuaLaTeX template – “Zeugnis”
%  Reads data from   student_data.lua   (see sample next to this file)
%  and produces a 2‑page A4 report similar to the HTML version.
% --------------------------------------------------------------

% --- Preamble -------------------------------------------------
%% ----------------------------------------------------
%  Minified, LuaLaTeX-friendly preamble for “Zeugnis”
% ----------------------------------------------------
\documentclass[12pt,a4paper]{article}

% ---------------- Page layout & colours -------------
\usepackage[top=2cm,bottom=2cm,
inner=1.5cm,outer=1.5cm]{geometry}

\usepackage[table]{xcolor}
\definecolor{egeblue}{RGB}{0,99,142}
\definecolor{egelightblue}{RGB}{66,144,179}
\definecolor{greenEnglish}{rgb}{0,0.5,0}

% ---------------- Language --------------------------
\usepackage[german]{babel}      % LuaLaTeX handles UTF-8 natively

% ---------------- Fonts -----------------------------
\usepackage{fontspec}           % modern font loader
\setmainfont{Latin Modern Sans}        % or any installed font
%\setsansfont{Helvetica Neue}            % replaces \usepackage{helvet}
%\setsansfont{Latin Modern Sans}
\usepackage{amsmath,amssymb}
%\usepackage{unicode-math}       % full Unicode maths
%\setmathfont{Latin Modern Math}

% If you need one single sans-serif default:
% \renewcommand*\familydefault{\sfdefault}

% ---------------- Micro-typography ------------------
\usepackage{microtype}          % protrusion & font expansion
\usepackage{selnolig}           % fixes spurious ligatures

% ---------------- Headings, spacing -----------------
\usepackage{titlesec}
\usepackage{setspace}           % \singlespacing, etc.
\usepackage{ragged2e}
\usepackage{enumitem}

% ---------------- Lua helpers -----------------------
\usepackage{luacode}

% ---------------- Graphics / header / footer --------
\usepackage{graphicx}
\usepackage{fancyhdr}

% ---------------- Tables & boxes --------------------
\usepackage{array,tabularx,xltabular}
%\setlength{\LTpre}{0pt}\setlength{\LTpost}{0pt}
\usepackage{cellspace}
\makeatletter
\let\@startpbox@action\@startpbox
\makeatother

\usepackage{makecell,colortbl,booktabs,multirow}

\usepackage{nowidow}            % prevents widows/orphans

% convenient column types
\newcolumntype{Y}{>{\centering\arraybackslash}X}
\newcolumntype{L}[1]{>{\raggedright\arraybackslash}p{#1}}
%\newcolumntype{C}[1]{>{\centering\arraybackslash}m{#1}}
\newcolumntype{R}[1]{>{\raggedleft\arraybackslash}p{#1}}
\newcolumntype{J}[1]{>{\justifying\arraybackslash}p{#1}}
\newcolumntype{v}[1]{>{\raggedright\hspace{0em}}p{#1}}
\newcolumntype{C}[1]{>{\centering\arraybackslash}m{#1}}
\renewcommand*\tabularxcolumn[1]{>{}m{#1}}

\newcommand*\rot{\rotatebox{90}}
%\renewcommand\theadalign{l}
\renewcommand\theadfont{\bfseries\footnotesize}

% ---------------- Maths & symbols -------------------
\usepackage{amsmath,amssymb}

% ---------------- Strikeout, underline --------------
\usepackage[normalem]{ulem}      % keep \emph italic, enable \sout

% ---------------- Misc helpers ----------------------
\usepackage{xstring,etoolbox}

% If you still need a *specific* Unicode symbol that your fonts
% don’t contain, you can define it once, e.g.:
% \newunicodechar{⟨}{\ensuremath\langle}

% ----------------------------------------------------
\setlength{\parindent}{0pt}
\newcommand{\tableGradeWidth}{.080\linewidth}
\newcommand{\tableGradeWidthLevel}{.06\linewidth}


% ----------------------------------------------------
%  Minified, LuaLaTeX-friendly preamble for “Zeugnis”
% ----------------------------------------------------
\documentclass[12pt,a4paper]{article}

% ---------------- Page layout & colours -------------
\usepackage[top=2cm,bottom=2cm,
inner=2cm,outer=2cm]{geometry}

\usepackage[table]{xcolor}
\definecolor{egeblue}{RGB}{0,99,142}
\definecolor{egelightblue}{RGB}{66,144,179}
\definecolor{greenEnglish}{rgb}{0,0.5,0}

% ---------------- Language --------------------------
\usepackage[german]{babel}      % LuaLaTeX handles UTF-8 natively

% ---------------- Fonts -----------------------------
\usepackage{fontspec}           % modern font loader
\setmainfont{Latin Modern Sans}        % or any installed font
%\setsansfont{Helvetica Neue}            % replaces \usepackage{helvet}
%\setsansfont{Latin Modern Sans}
\usepackage{amsmath,amssymb}
%\usepackage{unicode-math}       % full Unicode maths
%\setmathfont{Latin Modern Math}

% If you need one single sans-serif default:
% \renewcommand*\familydefault{\sfdefault}

% ---------------- Micro-typography ------------------
\usepackage{microtype}          % protrusion & font expansion
\usepackage{selnolig}           % fixes spurious ligatures

% ---------------- Headings, spacing -----------------
\usepackage{titlesec}
\usepackage{setspace}           % \singlespacing, etc.
\usepackage{ragged2e}
\usepackage{enumitem}

% ---------------- Lua helpers -----------------------
\usepackage{luacode}

% ---------------- Graphics / header / footer --------
\usepackage{graphicx}
\usepackage{fancyhdr}

% ---------------- Tables & boxes --------------------
\usepackage{array,tabularx,xltabular}
%\setlength{\LTpre}{0pt}\setlength{\LTpost}{0pt}
\usepackage[column=K]{cellspace}

\usepackage{makecell,colortbl,booktabs,multirow}

\usepackage{nowidow}            % prevents widows/orphans

% convenient column types
\newcolumntype{Y}{>{\centering\arraybackslash}X}
\newcolumntype{L}[1]{>{\raggedright\arraybackslash}p{#1}}
\newcolumntype{C}[1]{>{\centering\arraybackslash}p{#1}}
\newcolumntype{R}[1]{>{\raggedleft\arraybackslash}p{#1}}
\newcolumntype{J}[1]{>{\justifying\arraybackslash}p{#1}}
\newcolumntype{v}[1]{>{\raggedright\hspace{0em}}p{#1}}
\newcommand*\rot{\rotatebox{90}}
%\renewcommand\theadalign{l}
\renewcommand\theadfont{\bfseries\footnotesize}

% ---------------- Maths & symbols -------------------
\usepackage{amsmath,amssymb}

% ---------------- Strikeout, underline --------------
\usepackage[normalem]{ulem}      % keep \emph italic, enable \sout

% ---------------- Misc helpers ----------------------
\usepackage{xstring,etoolbox}

% If you still need a *specific* Unicode symbol that your fonts
% don’t contain, you can define it once, e.g.:
% \newunicodechar{⟨}{\ensuremath\langle}

% ----------------------------------------------------
\setlength{\parindent}{0pt}
\newcommand{\tableGradeWidth}{.080\linewidth}
\newcommand{\tableGradeWidthLevel}{.06\linewidth}


%\usepackage{cellspace}            % adds top/bottom padding
% -----------------------------------------------------------------
% 2 — vertical padding = 1.5 mm
\setlength\cellspacetoplimit{1.5mm}
\setlength\cellspacebottomlimit{1.5mm}
% Activate padding in *every* tabular-like env:
% -----------------------------------------------------------------
% 3 — column helpers
%\newcolumntype{Y}{>{\centering\arraybackslash}m{18mm}} % 18 mm wide, centred *vertically & horizontally*
\newcommand*\theadcell[1]{\makecell[c]{#1}}
\newcommand{\schoolReportDay}{27.06.2025}
%\newcommand{\myZeugnisTable}[2]{
	\begin{tblr}{
			width = \linewidth,
			colspec = {Q[l,m,wd=11cm] X[c] X[c] X[c] X[c]},
			hlines = {1pt,solid},
			vlines = {0.6pt,solid},
			rowsep = 2pt,
			colsep = 3pt,
			row{1} = {bg=egelightblue, fg=white, font=\bfseries\footnotesize},
		}
		\textbf{\large #1} & 
		\shortstack[c]{\textbf{sehr gut}\\ \textbf{erfüllt}} &
		\shortstack[c]{\textbf{gut}\\ \textbf{erfüllt}} &
		\shortstack[c]{\textbf{teilweise}\\ \textbf{erfüllt}} &
		\shortstack[c]{\textbf{nicht}\\ \textbf{erfüllt}} \\
		#2
	\end{tblr}
}

\makeatletter
\def\nobreakhline{%
	\noalign{\ifnum0=`}\fi
	\penalty\@M
	\futurelet\@let@token\LT@@nobreakhline}
\def\LT@@nobreakhline{%
	\ifx\@let@token\hline
	\global\let\@gtempa\@gobble
	\gdef\LT@sep{\penalty\@M\vskip\doublerulesep}% <-- change here
	\else
	\global\let\@gtempa\@empty
	\gdef\LT@sep{\penalty\@M\vskip-\arrayrulewidth}% <-- change here
	\fi
	\ifnum0=`{\fi}%
	\multispan\LT@cols
	\unskip\leaders\hrule\@height\arrayrulewidth\hfill\cr
	\noalign{\LT@sep}%
	\multispan\LT@cols
	\unskip\leaders\hrule\@height\arrayrulewidth\hfill\cr
	\noalign{\penalty\@M}%
	\@gtempa}
\makeatother


%\newcommand\competencyLevel{
%	\begin{tabularx}{\linewidth}{|X|C{\tableGradeWidth}|C{\tableGradeWidth}|C{\tableGradeWidth}|C{\tableGradeWidth}|}
%		\hline
%		\textbf{Kompetenzfelder und Lernziele der Fachbereiche} & \rot{\parbox{2.7cm}{\thead{sehr gut erfüllt}}} & \rot{\thead{gut erfüllt}} & \rot{\thead{teilweise}} & \rot{\thead{nicht erfüllt}} \\
%		\hline
%	\end{tabularx}
%}

\newcommand{\gradeOne}{& $\boxtimes$ & $\Box$ & $\Box$ & $\Box$\\}
\newcommand{\gradeTwo}{& $\Box$ & $\boxtimes$ & $\Box$ & $\Box$\\\nobreakhline}
\newcommand{\gradeThree}{& $\Box$ & $\Box$ & $\boxtimes$ & $\Box$\\\nobreakhline}
\newcommand{\gradeFour}{& $\Box$ & $\Box$ & $\Box$ & $\boxtimes$\\\nobreakhline}
\newcommand{\gradeNotGiven}{& \multicolumn{4}{c|}{nicht erteilt}\\\nobreakhline}
\newcommand{\gradeComesWithSecondHalfYear}{& \multicolumn{4}{c|}{wird im 2. Halbjahr belegt}\\\nobreakhline}
%\newcommand{\gradeDefault}{&  &  &  & \\\nobreakhline}
\newcommand\gradeDefault{& \multicolumn{4}{c|}{\textemdash} \\}
\newcommand{\gradeSkip}{& $\Box$ & $\Box$ & $\Box$ & $\Box$\\\nobreakhline}


\newcommand{\levelOne}{\centering Du hast vorwiegend auf {\color{greenEnglish} Anforderungsebene 1} gearbeitet.}
\newcommand{\levelTwo}{\centering Du hast vorwiegend auf {\color{blue} Anforderungsebene 2} gearbeitet.}
\newcommand{\levelThree}{\centering Du hast vorwiegend auf {\color{red} Anforderungsebene 3} gearbeitet.}
\newcommand{\levelSeven}{\centering bis Klasse 7 ohne Anforderungsebene}
\newcommand{\levelEight}{\centering bis Klasse 8 ohne Anforderungsebene}
\newcommand{\levelNine}{\centering bis Klasse 9 ohne Anforderungsebene}
\newcommand{\noLevel}{}
%\newcommand{\compText}[1]{\multicolumn{5}{|p{0.975\linewidth}|}{#1}\\\hline}


% argument is the page threshold until a newline should be added
\newcommand\newOptionalNewPage[1]{
	\ifnum #1>\value{page}
		\newpage
		\thispagestyle{plain}
		\phantom{~}
	\fi
}

\newcommand{\newpagedefs}{
	\newpage
	\newgeometry{headheight=90pt,top=7cm,
		bottom=2cm, 
		inner=2cm,
		outer=2cm}
	\setlength{\headsep}{-0.4cm}
	\pagestyle{fancy}
}

\newcommand{\newpagedefsLastPage}{
	\newpage
	\newgeometry{top=2cm,
		bottom=2cm, 
		inner=2cm,
		outer=2cm}
	\pagestyle{fancy}
}

\newcommand\formatText[1]{
	\noexpandarg
	\StrBefore{#1}{\\}[\studentIntro]
	\StrBehind{#1}{\\}[\certText]
	{\Large\setstretch{1.10}\textbf{\LARGE\studentIntro\vspace{.5em}\\}\nowidow[11]\noclub[11]\certText\par}
}


\pagestyle{fancy}

\fancypagestyle{mypagestyle}{%
	\chead{\competencyLevel}
	\cfoot{\small\raggedright\textit{Legende}:	Anforderungsebene Grün = Hauptschule ( AE I ), Anforderungsebene Blau = Realschule ( AE II ), Anforderungsebene Rot = Gymnasium ( AE III ),	n.b. = nicht bewertet}
}


\widowpenalty10000
\clubpenalty10000

% --- Lua helpers ----------------------------------------------
%\begin{luacode*}
%	-- Load student data table
%	dofile("student_data.lua")
%	
%	-- Escape TeX‑special characters coming from Lua strings
%	 function tex_escape(str)
%	 return (str
%		:gsub("([%%#$&{}_\\])", "\\%1")   -- escape % # $ & _ { } \
%		:gsub("\n", "\\\\"))              -- newline → \\
%  	 end
%	
%	-- Convenience for checkbox columns (1‑4)
%	function checkbox(level, col)
%	return (level == col) and "$\\boxtimes$" or "$\\Box$"
%	end
%\end{luacode*}

\newcommand{\headcol}[2]{\cellcolor{#1}\color{white}\bfseries #2}


% --- Document --------------------------------------------------
\begin{document}
	
	% ---------- Title page ----------------------------------------
%	\include{Zeugnis_titlepage.tex}
	\pagestyle{plain}
%\fcolorbox{egeblue}{white}{
%	\begin{minipage}[t][.98\textheight][t]{.97\textwidth}
%		\begin{center}\vspace{1cm}
%			\includegraphics[width=.9\linewidth]{Logotop_mit_ESM_zeugnis}
%		\end{center}
%		\vspace{3cm}
%		\begin{center}\setstretch{1.5}
%			\textbf{\Huge Zeugnis}\\
%			\textbf{\LARGE ~\\
%				Evangelische Gemeinschaftsschule Erfurt\\\vspace{2cm}
%				\directlua{tex.sprint(student.part_of_year)} \directlua{tex.sprint(student.school_year)}}
%		\end{center}\vspace{3cm}
%		\begin{center}\LARGE
%			\begin{tabular}{ll}
%				\textbf{Klasse:} &  \directlua{tex.sprint(student.classRoom)}\vspace{1cm}\\
%				\textbf{Name:} &   \directlua{tex.sprint(student.first_name)} \directlua{tex.sprint(student.last_name)}\\%
%				& geb.: \directlua{tex.sprint(student.date_of_birth)}\\%
%			\end{tabular}
%		\end{center}
%		\vfill
%		\begin{center}
%			\includegraphics[width=0.4\linewidth]{Stiftung}
%		\end{center}
%		\vspace{1cm}
%	\end{minipage}
%}%

	
	\newpage
	
	% ---------- Personal text -------------------------------------
%	Lieber \directlua{tex.sprint(student.first_name)},\\
%	\directlua{tex.sprint(student.personal_text)}
%	\newpage
%	\vspace{1em}
	
%	% ---------- Subjects & competences ----------------------------
%\begin{luacode*}	
%	for _, subj in ipairs(student.subjects) do
%		tex.print("\\section*{", tex_escape(subj.name), "}")		
%		for _, area in ipairs(subj.areas) do
%			tex.print("\\textbf{", tex_escape(area.title), "}\\\\\n")
%			
%			-- table header
%			tex.print("\\begin{longtable}{|p{9cm}|c|c|c|c|}\\hline\n",
%				"Kompetenz & Sehr gut & Gut & Teilweise & Nicht~erfüllt \\\\\\ \\hline\\endhead\n")
%				
%				for _, comp in ipairs(area.competences) do
%				local lvl = comp.level or 0                       -- nil-safe
%				tex.print(tex_escape(comp.description), " & ",
%				checkbox(lvl,1)," & ", checkbox(lvl,2)," & ",
%				checkbox(lvl,3)," & ", checkbox(lvl,4),
%				" \\\\\\ \\hline\n")
%			end
%			
%			tex.print("\\end{longtable}\n\\vspace{1em}\n")
%		end
%	end
%\end{luacode*}

%\begin{luacode*}	
%	for _, subj in ipairs(student.subjects) do
%		tex.print("\\renewcommand{\\arraystretch}{1.5}")
%		tex.print("\\begin{xltabular}{\\linewidth}{|X|C{\\tableGradeWidth}|C{\\tableGradeWidth}|C{\\tableGradeWidth}|C{\\tableGradeWidth}|}")
%			tex.print("\\nobreakhline")
%			tex.print("\\multicolumn{5}{|>{\\columncolor{egelightblue}}l|}{\\textit{\\textbf{" .. subj.name .. "}}}\\\\")
%			tex.print("\\nobreakhline")
%		for _, area in ipairs(subj.areas) do
%			for _, comp in ipairs(area.competences) do
%				local lvl = comp.level or 0                       -- nil-safe
%				tex.print(tex_escape(comp.description), " & ",
%				checkbox(lvl,1)," & ", checkbox(lvl,2)," & ",
%				checkbox(lvl,3)," & ", checkbox(lvl,4),
%				"\\\\")
%			end
%
%		end
%		tex.print("\\end{xltabular}")
%	end
%\end{luacode*}


%\begin{xltabular}{\textwidth}{|
%		>{\raggedright\arraybackslash\cellcolor{teal!70}\color{white}\bfseries}X
%		|Y|Y|Y|Y|}
%	\hline
%	Fach &
%	\theadcell{sehr gut\\erfüllt} &
%	\theadcell{gut\\erfüllt} &
%	\theadcell{teilweise\\erfüllt} &
%	\theadcell{nicht\\erfüllt} \\\hline%
%	%  …body of the table…
%	Deutsch      &   &   &   &   \\\hline%
%	Mathematik   &   &   &   &   \\\hline%
%\end{xltabular}



%		
%			tex.print("\\textbf{", tex_escape(area.title), "}\\\\\n")
%			
%			-- table header
%			tex.print("\\begin{longtable}{|p{9cm}|c|c|c|c|}\\hline\n",
%				"Kompetenz & Sehr gut & Gut & Teilweise & Nicht~erfüllt \\\\\\ \\hline\\endhead\n")
%				
%				
%			
%			tex.print("\\end{longtable}\n\\vspace{1em}\n")
%		end


%	\nobreakhline
%	#2
%	#3
%\end{xltabular}\vspace{2pt}


%	
%	% ---------- Participation / behaviour -------------------------

%\newpagedefsLastPage
\thispagestyle{plain}

\fcolorbox{egeblue}{white}{%
	\begin{minipage}[t][.99\textheight][t]{.97\textwidth}
		%		% ---------- remarks ---------------------------------------
		\textbf{Bemerkungen:}\par\vspace{.5em}%
		\directlua{tex.print(tex_escape(student.comment))}\par
		
		\vfill
		
		%		% ---------- absence block ---------------------------------
		\begin{tabularx}{\linewidth}{lX}
			Versäumnisse: 
			& \directlua{tex.sprint(student.absenceDaysTotal)}~Tage (davon \directlua{tex.sprint(student.absenceDaysUnauthorized)}~Tage unentschuldigt)\\[.3em]
			& \directlua{tex.sprint(student.absenceHoursTotal)}~Stunden (davon \directlua{tex.sprint(student.absenceHoursUnauthorized)}~Stunden unentschuldigt)\\%
		\end{tabularx}
		\vspace{3cm}\par
		Erfurt, den \schoolReportDay
		
		%		% ---------- signatures 1 ---------------------------------
		\vspace{2.5cm}\par
		\begin{tabularx}{\linewidth}{C{.4\linewidth}Y C{.4\linewidth}}
			\hrulefill & & \hrulefill \\[6pt]
			Klassenleiter/in & & Klassenleiter/in\\
		\end{tabularx}
		%		% ---------- signatures 2 ---------------------------------
		\vspace{2.5cm}\par
		\begin{tabularx}{\linewidth}{C{.4\linewidth}Y C{.4\linewidth}}
			\rule{\linewidth}{1pt} & & \\[-.7ex]
			{\centering SchulleiterIn} & {\tiny Siegel} & \\%
		\end{tabularx}		
		%		% ---------- signatures 3 ---------------------------------
		\vspace{2.5cm}\par
		\begin{tabularx}{\linewidth}{C{.4\linewidth}X C{.4\linewidth}}
			\rule{\linewidth}{1pt} & & \rule{\linewidth}{1pt}\\%
			{\centering Erziehungsberechtigte} & &	{\centering SchülerIn}\\%
		\end{tabularx}		
		\vfill		
		%		% ---------- legal notice ---------------------------------
		{\footnotesize
			\textbf{Rechtsbehelf:}\\
			Gegen die Versetzungsentscheidung kann innerhalb eines Monats nach
			Bekanntgabe des Zeugnisses Widerspruch erhoben werden. Der Widerspruch
			ist bei der Evangelischen Gemeinschaftsschule Erfurt,
			Eugen-Richter-Straße 22, 99085 Erfurt schriftlich oder zur Niederschrift
			zu erheben.\\[3em]
			\begin{tabularx}{\linewidth}{C{.6\linewidth}}
				\rule{\linewidth}{1pt} \\[-.3ex]
				{\centering Kenntnisnahme der Sorgeberechtigten}\\%
			\end{tabularx}
		}
	\end{minipage}%
}%
%\newpagedefsLastPage
\thispagestyle{plain}

\fcolorbox{egeblue}{white}{%
	\begin{minipage}[t][.99\textheight][t]{.97\textwidth}
%		% ---------- remarks ---------------------------------------
		\textbf{Bemerkungen:}\par\vspace{.5em}%
		\directlua{tex.print(tex_escape(student.comment))}\par
		
		\vfill
		
%		% ---------- absence block ---------------------------------
%		\begin{tabularx}{\linewidth}{lX}
%			Versäumnisse: &
%			\directlua{tex.print(student.absenceDaysTotal)}~Tage
%			(davon \directlua{tex.print(student.absenceDaysUnauthorized)}~Tage
%			unentschuldigt) \\[.3em]
%			& \directlua{tex.print(student.absenceHoursTotal)}~Stunden
%			(davon \directlua{tex.print(student.absenceHoursUnauthorized)}~Stunden
%			unentschuldigt)\\%
%		\end{tabularx}
		\vspace{3cm}\par
		Erfurt, den \schoolReportDay
%		
%		% ---------- signatures 1 ---------------------------------
		\vspace{2.5cm}\par
		\begin{tabularx}{\linewidth}{C{.4\linewidth} Y C{.4\linewidth}}
			\rule{\linewidth}{1pt} & & \rule{\linewidth}{1pt}\\%
			{\centering KlassenleiterIn} & & {\centering KlassenleiterIn}\\%
		\end{tabularx}		
%		% ---------- signatures 2 ---------------------------------
%		\vspace{2.5cm}\par
%		\begin{tabularx}{\linewidth}{C{.4\linewidth}Y C{.4\linewidth}}
%			\rule{\linewidth}{1pt} & & \\[-.7ex]
%			{\centering SchulleiterIn} & {\tiny Siegel} & \\%
%		\end{tabularx}		
%		% ---------- signatures 3 ---------------------------------
%		\vspace{2.5cm}\par
%		\begin{tabularx}{\linewidth}{C{.4\linewidth}X C{.4\linewidth}}
%			\rule{\linewidth}{1pt} & & \rule{\linewidth}{1pt}\\%
%			{\centering Erziehungsberechtigte} & &	{\centering SchülerIn}\\%
%		\end{tabularx}		
%		\vfill		
%		% ---------- legal notice ---------------------------------
%		{\footnotesize
%			\textbf{Rechtsbehelf:}\\
%			Gegen die Versetzungsentscheidung kann innerhalb eines Monats nach
%			Bekanntgabe des Zeugnisses Widerspruch erhoben werden. Der Widerspruch
%			ist bei der Evangelischen Gemeinschaftsschule Erfurt,
%			Eugen-Richter-Straße 22, 99085 Erfurt schriftlich oder zur Niederschrift
%			zu erheben.\\[3em]
%			\begin{tabularx}{\linewidth}{C{.6\linewidth}}
%				\rule{\linewidth}{1pt} \\[-.3ex]
%				{\centering Kenntnisnahme der Sorgeberechtigten}\\%
%			\end{tabularx}
%		}
	\end{minipage}%
}%
	
\end{document}