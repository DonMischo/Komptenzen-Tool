% --------------------------------------------------------------
%  Simple LuaLaTeX template – “Zeugnis”
%  Reads data from   student_data.lua   (see sample next to this file)
%  and produces a 2‑page A4 report similar to the HTML version.
% --------------------------------------------------------------

% --- Preamble -------------------------------------------------
% ----------------------------------------------------
%  Minified, LuaLaTeX-friendly preamble for “Zeugnis”
% ----------------------------------------------------
\documentclass[12pt,a4paper]{article}

% ---------------- Page layout & colours -------------
\usepackage[top=2cm,bottom=2cm,
inner=1.5cm,outer=1.5cm]{geometry}

\usepackage[table]{xcolor}
\definecolor{egeblue}{RGB}{0,99,142}
\definecolor{egelightblue}{RGB}{66,144,179}
\definecolor{greenEnglish}{rgb}{0,0.5,0}

% ---------------- Language --------------------------
\usepackage[german]{babel}      % LuaLaTeX handles UTF-8 natively

% ---------------- Fonts -----------------------------
\usepackage{fontspec}           % modern font loader
\setmainfont{Latin Modern Sans}        % or any installed font
%\setsansfont{Helvetica Neue}            % replaces \usepackage{helvet}
%\setsansfont{Latin Modern Sans}
\usepackage{amsmath,amssymb}
%\usepackage{unicode-math}       % full Unicode maths
%\setmathfont{Latin Modern Math}

% If you need one single sans-serif default:
% \renewcommand*\familydefault{\sfdefault}

% ---------------- Micro-typography ------------------
\usepackage{microtype}          % protrusion & font expansion
\usepackage{selnolig}           % fixes spurious ligatures

% ---------------- Headings, spacing -----------------
\usepackage{titlesec}
\usepackage{setspace}           % \singlespacing, etc.
\usepackage{ragged2e}
\usepackage{enumitem}

% ---------------- Lua helpers -----------------------
\usepackage{luacode}

% ---------------- Graphics / header / footer --------
\usepackage{graphicx}
\usepackage{fancyhdr}

% ---------------- Tables & boxes --------------------
\usepackage{array,tabularx,xltabular}
%\setlength{\LTpre}{0pt}\setlength{\LTpost}{0pt}
\usepackage{cellspace}
\makeatletter
\let\@startpbox@action\@startpbox
\makeatother

\usepackage{makecell,colortbl,booktabs,multirow}

\usepackage{nowidow}            % prevents widows/orphans

% convenient column types
\newcolumntype{Y}{>{\centering\arraybackslash}X}
\newcolumntype{L}[1]{>{\raggedright\arraybackslash}p{#1}}
%\newcolumntype{C}[1]{>{\centering\arraybackslash}m{#1}}
\newcolumntype{R}[1]{>{\raggedleft\arraybackslash}p{#1}}
\newcolumntype{J}[1]{>{\justifying\arraybackslash}p{#1}}
\newcolumntype{v}[1]{>{\raggedright\hspace{0em}}p{#1}}
\newcolumntype{C}[1]{>{\centering\arraybackslash}m{#1}}
\renewcommand*\tabularxcolumn[1]{>{}m{#1}}

\newcommand*\rot{\rotatebox{90}}
%\renewcommand\theadalign{l}
\renewcommand\theadfont{\bfseries\footnotesize}

% ---------------- Maths & symbols -------------------
\usepackage{amsmath,amssymb}

% ---------------- Strikeout, underline --------------
\usepackage[normalem]{ulem}      % keep \emph italic, enable \sout

% ---------------- Misc helpers ----------------------
\usepackage{xstring,etoolbox}

% If you still need a *specific* Unicode symbol that your fonts
% don’t contain, you can define it once, e.g.:
% \newunicodechar{⟨}{\ensuremath\langle}

% ----------------------------------------------------
\setlength{\parindent}{0pt}
\newcommand{\tableGradeWidth}{.080\linewidth}
\newcommand{\tableGradeWidthLevel}{.06\linewidth}

\newcommand{\myZeugnisTable}[2]{
	\begin{tblr}{
			width = \linewidth,
			colspec = {Q[l,m,wd=11cm] X[c] X[c] X[c] X[c]},
			hlines = {1pt,solid},
			vlines = {0.6pt,solid},
			rowsep = 2pt,
			colsep = 3pt,
			row{1} = {bg=egelightblue, fg=white, font=\bfseries\footnotesize},
		}
		\textbf{\large #1} & 
		\shortstack[c]{\textbf{sehr gut}\\ \textbf{erfüllt}} &
		\shortstack[c]{\textbf{gut}\\ \textbf{erfüllt}} &
		\shortstack[c]{\textbf{teilweise}\\ \textbf{erfüllt}} &
		\shortstack[c]{\textbf{nicht}\\ \textbf{erfüllt}} \\
		#2
	\end{tblr}
}

\makeatletter
\def\nobreakhline{%
	\noalign{\ifnum0=`}\fi
	\penalty\@M
	\futurelet\@let@token\LT@@nobreakhline}
\def\LT@@nobreakhline{%
	\ifx\@let@token\hline
	\global\let\@gtempa\@gobble
	\gdef\LT@sep{\penalty\@M\vskip\doublerulesep}% <-- change here
	\else
	\global\let\@gtempa\@empty
	\gdef\LT@sep{\penalty\@M\vskip-\arrayrulewidth}% <-- change here
	\fi
	\ifnum0=`{\fi}%
	\multispan\LT@cols
	\unskip\leaders\hrule\@height\arrayrulewidth\hfill\cr
	\noalign{\LT@sep}%
	\multispan\LT@cols
	\unskip\leaders\hrule\@height\arrayrulewidth\hfill\cr
	\noalign{\penalty\@M}%
	\@gtempa}
\makeatother


%\newcommand\competencyLevel{
%	\begin{tabularx}{\linewidth}{|X|C{\tableGradeWidth}|C{\tableGradeWidth}|C{\tableGradeWidth}|C{\tableGradeWidth}|}
%		\hline
%		\textbf{Kompetenzfelder und Lernziele der Fachbereiche} & \rot{\parbox{2.7cm}{\thead{sehr gut erfüllt}}} & \rot{\thead{gut erfüllt}} & \rot{\thead{teilweise}} & \rot{\thead{nicht erfüllt}} \\
%		\hline
%	\end{tabularx}
%}

\newcommand{\gradeOne}{& $\boxtimes$ & $\Box$ & $\Box$ & $\Box$\\}
\newcommand{\gradeTwo}{& $\Box$ & $\boxtimes$ & $\Box$ & $\Box$\\\nobreakhline}
\newcommand{\gradeThree}{& $\Box$ & $\Box$ & $\boxtimes$ & $\Box$\\\nobreakhline}
\newcommand{\gradeFour}{& $\Box$ & $\Box$ & $\Box$ & $\boxtimes$\\\nobreakhline}
\newcommand{\gradeNotGiven}{& \multicolumn{4}{c|}{nicht erteilt}\\\nobreakhline}
\newcommand{\gradeComesWithSecondHalfYear}{& \multicolumn{4}{c|}{wird im 2. Halbjahr belegt}\\\nobreakhline}
%\newcommand{\gradeDefault}{&  &  &  & \\\nobreakhline}
\newcommand\gradeDefault{& \multicolumn{4}{c|}{\textemdash} \\}
\newcommand{\gradeSkip}{& $\Box$ & $\Box$ & $\Box$ & $\Box$\\\nobreakhline}


\newcommand{\levelOne}{\centering Du hast vorwiegend auf {\color{greenEnglish} Anforderungsebene 1} gearbeitet.}
\newcommand{\levelTwo}{\centering Du hast vorwiegend auf {\color{blue} Anforderungsebene 2} gearbeitet.}
\newcommand{\levelThree}{\centering Du hast vorwiegend auf {\color{red} Anforderungsebene 3} gearbeitet.}
\newcommand{\levelSeven}{\centering bis Klasse 7 ohne Anforderungsebene}
\newcommand{\levelEight}{\centering bis Klasse 8 ohne Anforderungsebene}
\newcommand{\levelNine}{\centering bis Klasse 9 ohne Anforderungsebene}
\newcommand{\noLevel}{}
%\newcommand{\compText}[1]{\multicolumn{5}{|p{0.975\linewidth}|}{#1}\\\hline}


% argument is the page threshold until a newline should be added
\newcommand\newOptionalNewPage[1]{
	\ifnum #1>\value{page}
		\newpage
		\thispagestyle{plain}
		\phantom{~}
	\fi
}

\newcommand{\newpagedefs}{
	\newpage
	\newgeometry{headheight=90pt,top=7cm,
		bottom=2cm, 
		inner=2cm,
		outer=2cm}
	\setlength{\headsep}{-0.4cm}
	\pagestyle{fancy}
}

\newcommand{\newpagedefsLastPage}{
	\newpage
	\newgeometry{top=2cm,
		bottom=2cm, 
		inner=2cm,
		outer=2cm}
	\pagestyle{fancy}
}

\newcommand\formatText[1]{
	\noexpandarg
	\StrBefore{#1}{\\}[\studentIntro]
	\StrBehind{#1}{\\}[\certText]
	{\Large\setstretch{1.10}\textbf{\LARGE\studentIntro\vspace{.5em}\\}\nowidow[11]\noclub[11]\certText\par}
}


\pagestyle{fancy}

\fancypagestyle{mypagestyle}{%
	\chead{\competencyLevel}
	\cfoot{\small\raggedright\textit{Legende}:	Anforderungsebene Grün = Hauptschule ( AE I ), Anforderungsebene Blau = Realschule ( AE II ), Anforderungsebene Rot = Gymnasium ( AE III ),	n.b. = nicht bewertet}
}


\widowpenalty10000
\clubpenalty10000

% --- Lua helpers ----------------------------------------------
\begin{luacode*}
	-- Load student data table
	dofile("student_data.lua")
	
	-- Escape TeX‑special characters coming from Lua strings
	function tex_escape(str)
	return (str
	:gsub("([%%#$&{}_\\])", "\\%1")   -- escape % # $ & _ { } \
	:gsub("\n", "\\\\"))              -- newline → \\
	end
	
	-- Convenience for checkbox columns (1‑4)
	function checkbox(level, col)
	return (level == col) and "$\\boxtimes$" or "$\\Box$"
	end
\end{luacode*}

% --- Document --------------------------------------------------
\begin{document}
	
	% ---------- Title page ----------------------------------------
%	\include{Zeugnis_titlepage.tex}

	\newpage
	
	% ---------- Personal text -------------------------------------
%	Lieber \directlua{tex.sprint(student.first_name)},\\
%	\directlua{tex.sprint(student.personal_text)}
%	\newpage
%	\vspace{1em}
	
%	% ---------- Subjects & competences ----------------------------

%\begin{luacode*}
%	local gradeCmd = {

%for _, subj in ipairs(student.subjects) do
%	
%	-- ----------------------------------------------------------
%	-- 2. exactly one row per AREA  ➔ all competences in 1st cell
%	-- ----------------------------------------------------------
%    for _, area in ipairs(subj.areas) do
%		-- build the first (description) cell with makecell
%		local lines = { "\\textbf{" .. tex_escape(area.title) .. "}" }
%		for _, comp in ipairs(area.competences) do
%			table.insert(lines, tex_escape(comp.description))
%		end
%		local firstCell = "\\makecell[tl]{" .. table.concat(lines, "\\\\") .. "}"
%		
%		-- print the complete row
%		tex.print(firstCell .. " " .. (gradeCmd[area.grade] or "\\default"))
%	end
%	
%	tex.print("\\end{xltabular}")
%end
%\end{luacode*}


% -------------------------------------------------------------------
%  ▸ Lua-Code: erzeugt pro Fach eine \myZeugnisTable{…}{…}
% -------------------------------------------------------------------
\begin{luacode*}

%	if type(student) ~= "table" then
%	student = _G.student       -- fall back to global
%	end
	
	-- quick sanity check
	assert(type(student) == "table" and student.subjects,
	"student_data.lua must define a table 'student' with field 'subjects'!")
	
	

	local esc = { ["\\"]="\\textbackslash{}", ["%%"]="\\%",
		["_"]="\\_", ["{"]="\\{", ["}"]="\\}",
		["&"]="\\&",  ["#"]="\\#", ["^"]="\\textasciicircum{}",
		["~"]="\\textasciitilde{}" }
	local function tex_escape(str)  return (str:gsub("[\\%%_%{%}&#%^~]", esc)) end

	local gradeCmd = {
		[1]  = "& $\\boxtimes$ & $\\Box$      & $\\Box$      & $\\Box$\\\\",
		[2]  = "& $\\Box$      & $\\boxtimes$ & $\\Box$      & $\\Box$\\\\",
		[3]  = "& $\\Box$      & $\\Box$      & $\\boxtimes$ & $\\Box$\\\\",
		[4]  = "& $\\Box$      & $\\Box$      & $\\Box$      & $\\boxtimes$\\\\",
		ne   = "\\SetCell[c=5]{c} nicht erteilt\\\\",
		HJ2  = "\\SetCell[c=5]{c} wird im 2.~Halbjahr belegt\\\\",
		default = "\\SetCell[c=5]{c} ~\\\\"
	}

--	for _, subj in ipairs(student.subjects) do
--	tex.print("Hello World")
--	tex.print("\\myZeugnisTable{"..tex_escape(subj.name).."}{%")

		--	for _, topic in ipairs(subj.topics) do
		--	tex.print("\\textbf{"..tex_escape(topic.title).."}\\par%")
		--	for _, comp in ipairs(topic.competences) do
		--	tex.print(tex_escape(comp.description).."\\par"%)
		--	end
		--	tex.print( gradeCmd[topic.grade] or gradeCmd.default )
		--	end

		--	tex.print("\\SetCell[c=5]{c} rot \\\\")
			
		--	tex.print("}")
		--end
--	end
\end{luacode*}

%\newpagedefsLastPage
\thispagestyle{plain}

\fcolorbox{egeblue}{white}{%
	\begin{minipage}[t][.99\textheight][t]{.97\textwidth}
		%		% ---------- remarks ---------------------------------------
		\textbf{Bemerkungen:}\par\vspace{.5em}%
		\directlua{tex.print(tex_escape(student.comment))}\par
		
		\vfill
		
		%		% ---------- absence block ---------------------------------
		\begin{tabularx}{\linewidth}{lX}
			Versäumnisse: 
			& \directlua{tex.sprint(student.absenceDaysTotal)}~Tage (davon \directlua{tex.sprint(student.absenceDaysUnauthorized)}~Tage unentschuldigt)\\[.3em]
			& \directlua{tex.sprint(student.absenceHoursTotal)}~Stunden (davon \directlua{tex.sprint(student.absenceHoursUnauthorized)}~Stunden unentschuldigt)\\%
		\end{tabularx}
		\vspace{3cm}\par
		Erfurt, den \schoolReportDay
		
		%		% ---------- signatures 1 ---------------------------------
		\vspace{2.5cm}\par
		\begin{tabularx}{\linewidth}{C{.4\linewidth}Y C{.4\linewidth}}
			\hrulefill & & \hrulefill \\[6pt]
			Klassenleiter/in & & Klassenleiter/in\\
		\end{tabularx}
		%		% ---------- signatures 2 ---------------------------------
		\vspace{2.5cm}\par
		\begin{tabularx}{\linewidth}{C{.4\linewidth}Y C{.4\linewidth}}
			\rule{\linewidth}{1pt} & & \\[-.7ex]
			{\centering SchulleiterIn} & {\tiny Siegel} & \\%
		\end{tabularx}		
		%		% ---------- signatures 3 ---------------------------------
		\vspace{2.5cm}\par
		\begin{tabularx}{\linewidth}{C{.4\linewidth}X C{.4\linewidth}}
			\rule{\linewidth}{1pt} & & \rule{\linewidth}{1pt}\\%
			{\centering Erziehungsberechtigte} & &	{\centering SchülerIn}\\%
		\end{tabularx}		
		\vfill		
		%		% ---------- legal notice ---------------------------------
		{\footnotesize
			\textbf{Rechtsbehelf:}\\
			Gegen die Versetzungsentscheidung kann innerhalb eines Monats nach
			Bekanntgabe des Zeugnisses Widerspruch erhoben werden. Der Widerspruch
			ist bei der Evangelischen Gemeinschaftsschule Erfurt,
			Eugen-Richter-Straße 22, 99085 Erfurt schriftlich oder zur Niederschrift
			zu erheben.\\[3em]
			\begin{tabularx}{\linewidth}{C{.6\linewidth}}
				\rule{\linewidth}{1pt} \\[-.3ex]
				{\centering Kenntnisnahme der Sorgeberechtigten}\\%
			\end{tabularx}
		}
	\end{minipage}%
}%

	
\end{document}