\documentclass[12pt,a4paper]{article}
% --- Minimal package set for Zeugnis project -----------------
\usepackage[top=2cm,bottom=2cm,left=1.5cm,right=1.5cm]{geometry} % page layout

\usepackage[table]{xcolor}   % colours for cells, rules, etc.
\definecolor{egeblue}{RGB}{0,99,142}
\definecolor{egelightblue}{RGB}{66,144,179}
\definecolor{greenEnglish}{rgb}{0,0.5,0}

\usepackage{fontspec}        % LuaLaTeX/XeLaTeX font loader
\setmainfont{Latin Modern Sans}
\usepackage{amsmath,amssymb}

\usepackage{tabularray}      % modern table package used by \myZeugnisTable
\usepackage{makecell}

\usepackage{graphicx}        % needed by \includegraphics on the title page
\usepackage{microtype}       % optional but keeps line-breaking / kerning tidy

\usepackage{luacode}
\usepackage{nowidow}
\usepackage{setspace}
% -------------------------------------------------------------

\setlength{\parindent}{0pt}
\widowpenalty10000
\clubpenalty10000

% --- Commands for Zeugnis project -----------------
\newcommand{\myZeugnisTable}[2]{
	\begin{tblr}{
			width = \linewidth,
			colspec = {Q[l,m,wd=11cm] X[c] X[c] X[c] X[c]},
			hlines = {1pt,solid},
			vlines = {0.6pt,solid},
			rowsep = 2pt,
			colsep = 3pt,
			row{1} = {bg=egelightblue, fg=white, font=\bfseries\footnotesize},
		}
		\textbf{\large #1} & 
		\shortstack[c]{\textbf{sehr gut}\\ \textbf{erfüllt}} &
		\shortstack[c]{\textbf{gut}\\ \textbf{erfüllt}} &
		\shortstack[c]{\textbf{teilweise}\\ \textbf{erfüllt}} &
		\shortstack[c]{\textbf{nicht}\\ \textbf{erfüllt}} \\
		#2
	\end{tblr}
}

% ------------------------------------------------------------
%  Simple variant: 1-column header + one free-text row
% ------------------------------------------------------------
\newcommand{\myZeugnisTableSimple}[2]{
	\begin{tblr}{
			width = \linewidth,
			colspec = {X[l]},
			hlines = {1pt,solid},
			vlines = {0.6pt,solid},
			rowsep = 2pt,
			colsep = 3pt,
			row{1} = {bg=egelightblue, fg=white, font=\bfseries\footnotesize},
		}
		\textbf{\large #1}\\
		#2
	\end{tblr}
}

\begin{luacode*}
	-- Escape TeX‑special characters coming from Lua strings
	function tex_escape(str)
	return (str
	:gsub("([%%#$&{}_\\])", "\\%1")   -- escape % # $ & _ { } \
	:gsub("\n", "\\\\"))              -- newline → \\
	end
\end{luacode*}

% argument is the page threshold until a newline should be added
%%% Helper: create one completely blank, numbered page
\newcommand*\blankpage{%
	\newpage           % flush everything pending
	\null                % empty box → forces a page
	\thispagestyle{plain}% no headers/footers
}

%%% Pad the document up to a specified page number
\newcommand*\newOptionalNewPage[1]{%
	\begingroup
	\count0=\value{page}% scratch counter = current page
	\loop
	\ifnum\count0<#1   % while we are below the target …
	\blankpage       % … add a blank page
	\repeat
	\endgroup}
